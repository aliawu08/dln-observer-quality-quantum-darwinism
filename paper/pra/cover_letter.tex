\documentclass[11pt]{letter}
\usepackage[margin=1in]{geometry}
\usepackage{hyperref}

\signature{Alia Wu\\Risk Efficacy \& Redline Rising\\wut08@nyu.edu}
\date{\today}

\begin{document}
\begin{letter}{Editorial Office\\Physical Review A\\American Physical Society}

\opening{Dear Editors,}

I am submitting the manuscript ``Observer Quality as a Resource Variable in
Quantum Darwinism: Optimal Decoding, $\varepsilon$-Approximate Spectrum
Broadcast Structure, and a Central-Spin Worked Example'' for consideration
as a Regular Article in Physical Review~A.

This paper addresses a gap in the quantum Darwinism (QD) and spectrum
broadcast structure (SBS) literature: existing analyses typically treat
observers as ideal, with perfect access, noiseless readout, and unlimited
temporal horizons.  We introduce an explicit observer-quality triple
$(R_O,\Lambda_O,\tau_O)$ encoding access fraction, calibration noise, and
temporal horizon, and propagate these constraints through sample-complexity
bounds for pointer-value inference.  The main technical results are
Chernoff-type decoding bounds under calibration, a data-processing theorem
showing calibration cannot increase the quantum Chernoff exponent, and
$\varepsilon$-robust upgrades from ideal to approximate SBS with explicit
additive error control.  A central-spin pure-dephasing example demonstrates
the framework with quantities computed from Hamiltonian couplings and
depolarizing readout noise, including robustness checks against coupling
heterogeneity and access assumptions.

Two aspects of this work may be of particular interest to PRA readers.
First, the inverted sophistication result (Section~VI\,E) shows that an
unmonitored collective decoder can be strictly outperformed by product
decoding when coherence reliability falls below a critical threshold, with
the inversion requiring observer-side rather than system-side decoherence.
This is a counterintuitive consequence of the factor-of-two exponent gap
between collective and product measurements for pure-state binary
hypotheses.  Second, the paper introduces a decoder-stage classification
($q_D$/$q_L$/$q_N$) that connects observer measurement constraints to the
Dot/Linear/Network (DLN) framework developed in a companion preprint; the
PRA paper is self-contained and does not require familiarity with DLN.

The scope is deliberately restricted to binary pointer alphabets and
pure-state fragment records.  Mixed-state extensions and multi-hypothesis
generalizations are noted in the text but deferred to future work.

All numerical code and figure-generation scripts are publicly available at
\url{https://github.com/aliawu08/dln-observer-quality-quantum-darwinism}
and archived at Zenodo (DOI:~10.5281/zenodo.18610548).  This paper has not
been submitted elsewhere.

\textbf{Suggested referees.}  Given the paper's intersection of quantum
Darwinism, spectrum broadcast structure, and quantum hypothesis testing,
the following colleagues are well placed to evaluate it:
%
\begin{enumerate}
\item \textbf{Jaros\l{}aw K.~Korbicz} (University of Gda\'nsk) ---
    leading contributor to SBS theory and author of a recent comprehensive
    review of quantum Darwinism and SBS.
\item \textbf{Kamil Roszak} (Wroc\l{}aw University of Science and
    Technology) --- expert on entanglement and objectivity in pure-dephasing
    spin models, closely related to our central-spin worked example.
\item \textbf{Thao P.~Le} (University of Melbourne) --- contributor to the
    formal equivalence between strong quantum Darwinism and SBS.
\end{enumerate}

\closing{Sincerely,}

\end{letter}
\end{document}
